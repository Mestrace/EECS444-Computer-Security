\documentclass[11pt]{article}
\usepackage{amsmath,amssymb,amsfonts,amsthm}
\newcommand{\numpy}{{\tt numpy}}    % tt font for numpy
\usepackage{graphicx}
\usepackage{xcolor}
\usepackage{listings}

\definecolor{mGreen}{rgb}{0,0.6,0}
\definecolor{mGray}{rgb}{0.5,0.5,0.5}
\definecolor{mPurple}{rgb}{0.58,0,0.82}
\definecolor{backgroundColour}{rgb}{0.95,0.95,0.92}


\lstdefinestyle{CStyle}{
    backgroundcolor=\color{backgroundColour},   
    commentstyle=\color{mGreen},
    keywordstyle=\color{magenta},
    numberstyle=\tiny\color{mGray},
    stringstyle=\color{mPurple},
    basicstyle=\ttfamily\footnotesize,
    breakatwhitespace=false,         
    breaklines=true,                 
    captionpos=b,                    
    keepspaces=true,                 
    numbers=left,                    
    numbersep=5pt,                  
    showspaces=false,                
    showstringspaces=false,
    showtabs=false,                  
    tabsize=2,
    language=C,
}

\lstset{
xleftmargin=.1\textwidth, xrightmargin=.1\textwidth
}


\topmargin -.5in
\textheight 9in
\oddsidemargin -.25in
\evensidemargin -.25in
\textwidth 7in

\begin{document}


% ========== Edit your name here
\author{Yida Liu}
\title{EECS 444 Homework 3 Part 1}
\maketitle

\section{Crackme}

We change the equality condition in line 00401243 from "JE" to "JNE", which will bypass the conditions and jump to the success dialog.

\begin{lstlisting}[language={[x86masm]Assembler}]
00401243     75 07          JNE SHORT CRACKME.0040124C ; Original JE
00401245   . E8 18010000    CALL CRACKME.00401362
0040124A   .^EB 9A          JMP SHORT CRACKME.004011E6
0040124C   > E8 FC000000    CALL CRACKME.0040134D ; jump to success dialog
\end{lstlisting}

\section{Crackme 2}

By analyzing the assembly, we realize that the original code performs the following, for a given name
\begin{enumerate}
    \item Sum the uppercase ASCII value of the input names.
    \item XOR 0x5678
    \item XOR 0x1234
\end{enumerate}
After this, a number is generated as the password.

\end{document}